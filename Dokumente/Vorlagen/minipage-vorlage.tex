%------------------------------
% Beispiel für eine Abbildung mit 2 nebeneinanderliegenden Graphen,
% beide mit einer Bildunterschrift und einer gemeinsamen Abbildungsbeschriftung
%
%------------------------------
%
\documentclass{scrartcl}
\usepackage{epic}
\usepackage{eepic}
\begin{document}
	%Begin der Abbildung
	\begin{figure}[htbp] %htbp setzt die Reihenfolge, wo die Grafik gesetzt werden soll 
		\vspace{0pt} %Abstand der Abbildung zum Text darüber
		\centering
		\begin{minipage}{6 cm}  %Breite des ersten Graphenbildes,
			 					%ist diese Breite Größer als der Graph entsteht ein Abstand zwischen den Graphen
			%erster Graphen hier einfügen (\unitlength und die gesamte picture Umgebung)
			\centering\mbox{Graph 1} %Untertitel des ersten Graphen (zentriert unter dem Bild)
			\label{graph1:label} %Label des ersten Graphen um darauf zu referenzieren
		\end{minipage}
		\begin{minipage}{6 cm} %Breite des zweiten Graphenbildes
			%zweiter Graph analog zum ersten
			\centering\mbox{Graph 2}
			\label{graph2:label}
		\end{minipage}
		\vspace{2cm}
		\caption{Ein Vergleich der Graphen vor und nach dem Algorithmus} %Abbildungstext unter beiden Graphen
	\end{figure}
\end{document}
